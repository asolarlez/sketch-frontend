\definecolor{light-gray}{gray}{0.6}

\newcommand{\SK}{\Sketch}
\newcommand{\Sk}{\Sketch}
\newcommand{\Sketch}{\textsc{Sketch}\xspace}

\newcommand{\TODO}[1]{[\textsl{#1}]}
\newcommand{\comment}[1]{[\textsl{#1}]}
\newcommand{\hide}[1]{} 
\newcommand{\C}[1]{\lstinline!#1!}
\newcommand {\spc}{\hspace{3pt}}
\newcommand {\lglg}{\textsc{L}og\textsc{L}og}
\newcommand {\llglg}{$\lambda$-\textsc{L}og\textsc{L}og}
\newcommand {\loglog}{\lglg}
%%% Inference rule things.
\newcommand{\deriv}[3]{\langle #1 \rangle #2 \langle #3 \rangle}
\newcommand{\initsenv}[1]{\Pi, \Lambda{#1}, \Sigma{#1}, \Gamma{#1}}
\newcommand{\expeval}{\stackrel{e}{\rightarrow}}

%%%%%%%%%%%%%%%%%%%%%%%%%%%%%%%%%%%%%%%%%%%
% Macros for the semantics
%%%%%%%%%%%%%%%%%%%%%%%%%%%%%%%%%%%%%%%%%%%

\newcommand{\valEnv}{\Sigma}
\newcommand{\typeEnv}{\Gamma}
\newcommand{\cdeclEnv}{\Pi}
\newcommand{\cEnv}{\Lambda}

%%%%%%%%%%%%%%%%%%%%%%%%%%%%%%%%%%%%%%%%%%%
%%%%%%%%%%%%%%%%%%%%%%%%%%%%%%%%%%%%%%%%%%%




  
\newcommand{\etal}{\textit{et al.}\@\xspace}
\newcommand{\eg}{\textit{e.g.}\@\xspace}
\newcommand{\ie}{\textit{i.e.}\@\xspace}


% References
%
\newtheorem{thm}{Theorem}
\newtheorem{lem}{Lemma}
\newtheorem{prop}{Property}

\newcommand{\thmlabel}[1]{\label{thm:#1}}
\newcommand{\thmref}[1]{Theorem~\ref{thm:#1}}
\newcommand{\lemlabel}[1]{\label{lem:#1}}
\newcommand{\lemref}[1]{Lemma~\ref{lem:#1}}

\newcommand{\corlabel}[1]{\label{cor:#1}}
\newcommand{\corref}[1]{Corollary~\ref{cor:#1}}

\newcommand{\proplabel}[1]{\label{prop:#1}}
\newcommand{\propref}[1]{Proposition~\ref{prop:#1}}
\newcommand{\deflabel}[1]{\label{def:#1}}
\newcommand{\defref}[1]{Definition~\ref{def:#1}}
\newcommand{\exlabel}[1]{\label{ex:#1}}
\newcommand{\exref}[1]{Example~\ref{ex:#1}}
\newcommand{\problabel}[1]{\label{prob:#1}}
\newcommand{\probref}[1]{Problem~\ref{prob:#1}}
\newcommand{\obslabel}[1]{\label{obs:#1}}
\newcommand{\obsref}[1]{Observation~\ref{obs:#1}}
\newcommand{\alglabel}[1]{\label{alg:#1}}
\newcommand{\algref}[1]{Algorithm~\ref{alg:#1}}
%
\newcommand{\applabel}[1]{\label{app:#1}}
\newcommand{\appref}[1]{Appendix~\ref{app:#1}}
\newcommand{\seclabel}[1]{\label{sec:#1}}
\newcommand{\shortsecref}[1]{\S\ref{sec:#1}}
\newcommand{\longsecref}[1]{Section~\ref{sec:#1}}
%
\newcommand{\tablabel}[1]{\label{tab:#1}}
\newcommand{\tabref}[1]{Table~\ref{tab:#1}}
\newcommand{\figlabel}[1]{\label{fig:#1}}
\newcommand{\longfigref}[1]{Figure~\ref{fig:#1}}
\newcommand{\shortfigref}[1]{Fig.~\ref{fig:#1}}
\newcommand{\eqqlabel}[1]{\label{eq:#1}}
\newcommand{\shorteqqref}[1]{\eqref{eq:#1}}
\newcommand{\mediumeqqref}[1]{Eq.~\eqref{eq:#1}}
\newcommand{\longeqqref}[1]{Equation~\eqref{eq:#1}}

% Determine type of references for particular document.
%
\newcommand{\secref}{\longsecref}
\newcommand{\figref}{\longfigref}
\newcommand{\eqqref}{\longeqqref}

% Numbering layout
%
\numberwithin{equation}{section}
\newtheorem{definition}{Definition}
\newtheorem{lemma}{Lemma}
\newtheorem{corolary}{Corollary}
\newtheorem{hypothesis}{Hypothesis}
\newtheorem{algo}{Algorithm}
\newtheorem{Equation}{Equation}
\newtheorem{Example}{Example}
\newtheorem{theorem}{Theorem}[section]
%\newenvironment{flagdoc}[1]{\begin{it}\paragraph{Flag \C{--#1}}}{\end{it}}

\newcommand{\flagdoc}[2]{ \newglossaryentry{#1}{name={\texttt{-$ $-#1}}, description={#2}} \begin{it}\paragraph{Flag \gls{#1}}#2\end{it} }
\newcommand{\flagdocb}[2]{ \newglossaryentry{#1}{name={\texttt{-#1}}, description={#2}} \begin{it}\paragraph{Flag \gls{#1}}#2\end{it} }



                     %%% Commands for formatting code %%%

% \code -- used for names of functions and other code
% Example:  This line calls the \code{numerator} method.
\newcommand{\code}[1]{\lstinline{#1}}  % This _must_ be {\tt#1}, not \texttt{#1}

              %%% Commands for formatting the language grammar %%%

% \token -- used for a literal code token
% \nterm -- used for a grammar non-terminal
% Example:  Every \nterm{AssertStmt} begins with the \token{assert} keyword.
\newcommand{\token}[1]{\textbf{#1}}
\newcommand{\formatnt}[1]{{\sl#1}}  % This _must_ be {\sl#1}, not \textsl{#1}
\newcommand{\nterm}[1]{\index{#1@\formatnt{#1}}\formatnt{#1}}

% syntax -- environment for specifiying syntax
\newenvironment{syntax}
 {\par\begin{tabular}{rcl}}
 {\end{tabular}\vspace{2ex}}

% \lexrule -- used to define a lexical regexp rule within a syntax environment
% Example:  \lexrule{IntegerLiteral}{[0-9]+}
\newcommand{\lexrule}[2]
 {\index{#1@\formatnt{#1}|defpage}\formatnt{#1} &
  $=$ & $\langle${\tt#2}$\rangle$ \\}

% \grammar -- used to define a grammar non-terminal within a syntax environment
% \grammaralt -- used for alternate definitions within a syntax environment
% Example:  \grammar{Expr}{\nterm{Expr} \token{+} \nterm{Expr}}
%           \grammaralt{\token{(} \nterm{Expr} \token{)}}
\newcommand{\grammar}[2]
 {\index{#1@\formatnt{#1}|defpage}\formatnt{#1} & $=$ & {#2} \\}
\newcommand{\grammaralt}[1]{& $|$ & {#1} \\}
\newcommand{\grammarelt}[2]
 {\index{#1@\formatnt{#1}|defpage}\formatnt{#1} & $\in$ & {#2} \\}

\newcommand{\galt}{\mbox{\hspace{0.7em}\ensuremath{|}\hspace{1em}}}

%%% Commands for inserting special characters %%%

% \bs -- used to create a monospace backslash
\newcommand{\bs}{{\tt\char"5C}}

% \us -- used to create a monospace underscore (works better than {\tt\_})
\newcommand{\us}{{\tt\char"5F}}


\usepackage[T1]{fontenc}
%\usepackage[scaled=0.85]{luximono}

\newcommand{\FILTER}{filter}
\usepackage{listings}
\usepackage[T1]{fontenc}
\usepackage[scaled=0.8]{luximono}
\lstdefinelanguage{sketch}{%
 morekeywords={%
  bit,boolean,implements,for,break,int,char,%
  while,if,else,foreach,until,return,false,true,%
  loop,double,static,void,float,insert,into,%
  reorder, repeat, null,class,fork,atomic,lock,unlock,struct,%
  %% following keywords are for the formal language that we use to model SKETCH
  def, defgen, generator, let,do,then,skip,op,assert,skip,switch,case,new
 },
 morecomment=[l]{//},
 otherkeywords={??,\{|,|\}},
 mathescape=true,
}
\lstset{basicstyle=\tt,language=sketch}